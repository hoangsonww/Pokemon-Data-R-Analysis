\documentclass[11pt]{article}
\usepackage[margin=1in]{geometry}
\usepackage{graphicx}
\usepackage{caption}
\usepackage{hyperref}

\title{Pokémon Attack Analysis Report}
\author{Your Name}
\date{\today}

\begin{document}
\maketitle

\begin{abstract}
An exploratory analysis of the "Attack" statistic across 1,008 Pokémon.  
Includes distribution, empirical rule checks, sampling behavior, and normality assessment.
\end{abstract}

\section{Data Overview}
We analyze the `Attack` values from a dataset of 1,008 Pokémon.  
Summary statistics:
\begin{itemize}
  \item Population mean: \(\mu \approx 77.17\)  
  \item Population SD: \(\sigma \approx 29.76\)  
\end{itemize}

\section{Population Distribution}
\begin{figure}[h!]
  \centering
  \includegraphics[width=0.8\textwidth]{attack-population.png}
  \caption{Histogram of all Pokémon Attack values (binwidth=10).}
\end{figure}

\section{Empirical Rule Checks}
\begin{itemize}
  \item Within 1 SD: 669 Pokémon (≈ 66.3\%)  
  \item Within 2 SD: 980 Pokémon (≈ 97.2\%)  
  \item Within 3 SD: 1,007 Pokémon (≈ 99.9\%)  
\end{itemize}

\section{Sampling Experiment}
A random sample of 200 Attack values (with replacement) was drawn.  
Sample summary:
\begin{itemize}
  \item Sample mean: varies per draw  
  \item Sample SD: varies per draw  
\end{itemize}

\begin{figure}[h!]
  \centering
  \includegraphics[width=0.8\textwidth]{attack-sample.png}
  \caption{Histogram of sampled Attack values (binwidth=10).}
\end{figure}

\section{Conclusion}
The population distribution is right-skewed, deviating slightly from normal.  
A sample of 200 shows noticeable sampling variability—larger samples (e.g.\ $\ge600$) improve representativeness.

\end{document}
